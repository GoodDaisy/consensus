\documentclass{article}
%
\usepackage{amsmath} 
\usepackage{amssymb}
\usepackage{amsthm}
\usepackage{fullpage}

\usepackage[bookmarks=true, unicode=true, pdftitle={GRANDPA: a blockchain finality gadget}, pdfauthor={Alistair Stewart},pdfkeywords={Polkadot, blockchain finality gadget},pdfborder={0 0 0.5 [1 3]}]{hyperref}
\usepackage{url}
\usepackage[numbers]{natbib}

\bibliographystyle{plainnat}

\newtheorem{theorem}{Theorem}[section]
\newtheorem{definition}[theorem]{Definition}
\newtheorem{lemma}[theorem]{Lemma}
\newtheorem{corollary}[theorem]{Corollary}
\newtheorem{proposition}[theorem]{Proposition}

\begin{document}


\title{Polkadot Finality Gadget v9000}
\date{\today}
\maketitle

\section{Introduction}

We consider the question of finality for blockchain protocols: when will a block be reverted. Many such protocols, such as the original blockchain, Bitcoin, have the property of eventual consensus - that an ever growing prefix of the chain will be agreed upon by all participants forever onward.
But they generally only give probabilistic finality on a specific block - that under some assumptions about the network and participants, if we see a few blocks building on a given block, we can estimate the probability that it is final.

But what we'd prefer is to have provable finality - for example a signed statement by a set of authorities, the set of whom can be tracked, that the block is final.
This is useful to prove what happened to light clients, who do not have the full chain or are not actively listening to the network, and to communicate with other chains, possibly as part of a scalability solution, where not anyone receives or stores all the data in the system. 

Another popular consensus mechanism for blockchains is to get Byzantine agreement on each block.
This gives provable finality immediately. However this is slow if we have a large set of participants in the Byzantine agreement.

The approach that we will take is similar to the approach that Ethereum plans to take with Casper the Friendly Finality Gadget (Casper FFG)\cite{CasperFFG}, which combines these approaches.
We will use a block production mechanism and chain selection rule that give eventual consensus and then add a finality gadget, a protocol that finalises blocks that the participants already agree on, to get provable finality.

We present a finality gadget that works in a partially synchronous network model, GRANDPA, as well as an asynchronous finality gadget, that can cope with $1/5$ Byzantine nodes. We first seek to formalise the finality gadget problem.

\subsection{Formalising the problem}

We need to incorporate into the definition of Byzantine agreement that we have access to a protocol that would achieve eventual consensus if we did not affect it. Consider a typical definition of a multi-values Byzantine agreement: 
We have a set of participants $V$, the majority of whom obey the protocol, but a constant fraction may be Byzantine, meaning they behave arbitrarily, e.g. provide false or inconsistent information or randomly go offline when they ought to be online.

\begin{definition} A protocol for {\em multi-valued Byzantine agreement} has a set of values $S$ and a set of voters $V$, a constant fraction of which may be Byzantine, for which each voter $v \in V$ starts with an initial value $s_v \in S$ and, in the end, decides a final value $f_v \in S$ such that the following holds:

\begin{itemize}
\item {\bf Agreement}: All honest voters decide the same value for $f_v$
\item {\bf Termination}: All honest voters eventually decide a value
\item {\bf Validity}: If all honest voters have the same initial value, then they all decide that value
\end{itemize}

\end{definition}

We can change this definition to assume that instead of having an initial value, all voters have access to an external protocol, an oracle for values, that achieves eventual consensus in that it returns the same value to all voters when called after some time.

\begin{definition} A protocol for the {\em multi-valued Byzantine finality gadget problem} has a set of values $S$, a set of voters $V$, a constant fraction of which may be Byzantine, for which each voter $v \in V$ has access to an oracle $A$ with the property that ??? in the end each voter decides a final value $f_v \in S$ such that the following holds:

\begin{itemize}
\item {\bf Agreement:} All honest voters decide the same value for $f_v$
\item {\bf Termination:} All honest voters eventually decide a value
\item {\bf Validity:} All honest voters decide a value that $A$ returned to some honest voter sometime.
\end{itemize}

\end{definition}

In the case where $|S| > 2$, this definition of validity is stronger than the validity notion in multi-valued Byzantine agreement ported here verbatim because all honest voters decide a value with which some honest voter started. 
This is because our earlier definition would be impossible if the fraction of Byzantine voters exceeds $1/|S|$, as we cannot detect Byzantine voters who act like honest voters, except for lying about their initial value. So if fewer than $1/|S|$ voters act like they have some initial value, the protocol cannot know if any are honest. 

But for the case $|S|=2$, the two possible definitions of validity are equivalent. This means that we can reduce the binary version of the Byzantine finality gadget problem above to binary Byzantine agreement, by each voter just calling $A$ at the start to obtain their initial value, since if $A$ does not return the same value to every honest voter all the time, then it returns both values to honest voters some times. Thus there are many existing algorithms for the binary Byzantine finality gadget problem. However the interesting problem in this case is whether the celebrated impossibility result of \cite{flp} generalizes to this finality gadget problem, i.e., whether this oracle which is guaranteed to achieve eventual consensus makes it possible to have an asynchronous and deterministic protocol for agreement. A reduction is not immediately obvious. It turns out that the finality gadget version is indeed impossible, as we shall see in \ref{ssec:impossibility}.

Now how do we extend this to agreeing on a chain of blocks? We will need the block production mechanism to build on finalised blocks, so the best chain rule must include them. We assume a kind of conditional eventual consensus. If we keep building on our last finalised block $B$ and don't finalise any new blocks, then eventually we have consensus on a longer chain than just $B$, which the finality gadget can use to finalise another block. We also want a protocol that does not terminate, but instead keeps on finalising more blocks. 

\begin{definition} A protocol for the {\em blockchain Byzantine finality gadget problem} has a set of voters $V$, a constant fraction of which may be Byzantine, for which each voter $v \in V$ has access to an oracle for the best chain given the last finalised block with the property that, as long as no new block is finalised, it achieves eventual consensus on some child of the last finalised block  such that the following holds:

\begin{itemize}
\item{\bf Safety:} All honest voters finalise the same block at each block number.
\item{\bf Liveness:} All honest voters keep finalising blocks.
\item{\bf Validity:} If an honest voter finalises a block $B$ then that block was seen in the best chain observed by some honest voter containing some previously finalised ancestor of $B$,
\end{itemize}

\end{definition}

We also consider temporal versions of the last two properties:

\begin{itemize}
\item{\bf Fast termination:} If the last finalised block has number $n$ and, until another block is finalised, the best chain observed by all validators will include the same block with block number $n+1$, then a block with number $n+1$ will be finalised within time $T$.
\item{\bf Recent validity:} If an honest voter finalises a block $B$ then that block was seen in the best chain observed by some honest voter containing some previously finalised ancestor of $B$ more recently than time $T$ ago.
\end{itemize}
These will typically only hold with high probability. In the asynchronous case, we would need to measure time in rounds of the protocol rather than seconds to make sense of these properties.

Lastly we are interested in the property of {\bf accountable safety}. This is that, if there are more than $f+1$ validators and blocks on different chains are finalised, then we can identify at least $f+1$ Byzantine validators. This wil hopefully allow us to remove them from the validator set and punish them. 

\subsection{Our approach}

To come up with a solution to the blockchain Byzantine finality gadget problem, we will typically look at various Byzantine agreement protocols and use those to find protocols for the  multi-valued Byzantine finality gadget problem. Protocols for that with appropriate properties can used to find protocols for the blockchain Byzantine finality gadget problem by considering running them in parallel at every block number. If the one block protocol has the right properties then they will agree on blocks consistently so if we finalise a block then we also finalise its ancestors and we can come up with a succinct protocol.

For example, suppose we have a one block protocol that calls for a vote on blocks which requires a participant to observe a supermajority, say votes from  $2/3$ of voters, for some block (or else the participant observes that the vote is undecided). Now imagine running this vote in parallel for every block number and have any honest voter vote for blocks from a particular chain. Byzantine voters may vote more than once, but if we count a vote for a block as a vote for each ancestor of the block in the vote for the instance of the one block protocol with its number, then Byzantine voters must also vote for chains, though they can vote for multiple chains. If we do this, then we see that if a block has a supermajority in a vote, then so does all its ancestors in their votes. Thus the blocks with a supermajority form a chain. Furthermore, if only $1/3$ of voters equivocate then if a participant sees a subset of the votes for chains, then they must see a prefix of the chain of blocks that all the votes have supermajorities for. Intuitively, the protocol can agree on the prefix that $2/3$ of voters agree on using this.

To ensure safety, each participant maintains an estimate $E_r$ of the last block that could have been finalised in a round $r$. This has the property that in  future rounds it overestimates the block that could have been finalised so that in round $r$, the chain with head $E_{r-1}$ contains all blocks that could have been finalised. Any honest voter only votes in round $r$ for chains containing their estimate $E_{r-1}$ and this guarantees that any block that could have been finalised in round $r-1$ will be finalised in round $r$.

\subsection{Related Work}

\subsubsection{Comparison with Casper}

The concept of finality gadget was introduced in Casper the friendly finality gadget and this remains the finality gadget which is most similar to ours. So it makes sense to compare these. However first, we should mention the other protocols that are also called Casper.

The first Casper was Casper TFG. Casper CBC\cite{CasperCBC} gives a recent and clearly specified version of this protocol. It's fork choice rule uses the GHOST selection rule on votes. In Casper TFG, votes are blocks, but they are counted by participants (proposers and validators) like votes, which differs from how GHOST would be used with proof of work. It also has a flexible way of subjectively finalising blocks based on graphs of votes. 

In Casper FFG\cite{CasperFFG}, validators vote on links between checkpoints, which occur at block numbers that are multiples of, say, 50. If there are 2/3 votes for one block at consecutive checkpoints, then we can finalise a chain of blocks up to the first checkpoint.

Epochless Casper, 

Casper...

There are two main differences between Casper FFG and GRANDPA. One is that in GRANDPA, different voters can cast votes simultaneously for blocks at different heights.

The other main difference is how the finality gadget affects the fork-choice rule of the underlying block production mechanism. In GRANDPA, by default we will assume that this is only affected by having to include any finalised blocks. 
\cite{CasperFFG} does not specify a fork-choice rule, but it requires that we build on justified blocks for liveness. Later specifications of Casper use the GHOST rule on votes for fork-choice.

Only depending on finalised blocks gives a clearer separation between the block production mechanism and finality gadget. It may therefore be easier to adapt to other types of protocol that achieve eventual consensus—and there have been many diverse protocols that do this developed in the last few years. It also makes it far easier to prove liveness properties. If the finality gadget has not finalised anything and so does not interfere, then the underlying mechanism should reach eventual consensus, which should be enough for the finality gadget to finalise whatever we have consensus on.

On the other hand, while building on the longest chain in the absence of a finality gadget to maximize block rewards may be rational if everyone else does, this is not always the case for building on the longest chain including the last finalised block. This is because it may be likely that a different chain is going to be finalised, in which case the rational thing to do might be to build on that. The GHOST on votes fork choice rule of ? and ? may be more rational. It is not clear that it is, nor is it clear how to prove liveness for such a rule. Further research may be needed to show that there is a fork choice rule which is rational and leads to liveness for the finality gadget. 




\section{Preliminaries} \label{sec:prelims}

{\bf Network model}: We will be mostly using a partially synchronous gossip network model, such as that described in \cite{Tendermint} II A. Participants communicate via a gossip network, where they are connected to a subset of other participants, and forward all messages they receive to all their connected peers. We assume that the network graph is such that any Byzantine participants are not able to cut off an honest participant and so any message sent or received by an honest participant reaches all honest participants. The partial synchrony we will use is the model where messages are received within time $T$, but possibly only after some Global Synchronisation Time $GST$. Concretely, any message sent or received by some honest participant at time $t$ is received by all honest participants by time $GST+T$ at the latest.


{\bf voters}: We will want to change the set of participants who actively agree sometimes. To model this, we have a large set of participants who follow messages. For each voting step, there is a set of $n$ voters. We will frequently need to assume that for each such step, at most  $f < n/3$ voters are Byzantine. We need $n-f$ of voters to agree on finality. Whether or not block producers ever vote, they will need to be participants who track the state of the protocol.

{\bf votes}: A vote is a block hash, together with some metadata such as round number and type of vote (prevote/precommit), all signed with a voter's private key.

{\bf Rounds}: each participant has their own idea of what the current round number is. Every prevote and precommit has an associated round number. Honest voters only vote once (for each type of vote) in each round and don't vote in earlier rounds after later ones.

Participants remember which block they see as currently being the latest finalised block and an estimate of which block could have been finalised in the last round.




For block $B$, we write $\mathrm{chain}(B)$ for the chain whose head is $B$. The block number, $n(B)$ of a block $B$ is the length of $\mathrm{chain}(B)$.

For blocks $B'$, $B$, $B$ is later than $B'$ if it has a higher block number.
We write $B > B'$ or that $B$ is descendant of $B'$ for $B$, $B'$ appearing in the same blockchain with $B'$ later i.e. $B \in \mathrm{chain}(B')$ with $n(B') > n(B)$ and $B < B'$ or $B$ is an ancestor of $B'$ for $B' \in \mathrm{chain}(B)$ with $n(B) > n(B')$. $B \geq B'$ and $B \leq B'$ are similar except allowing $B = B$. We write $B \sim B'$ or $B$ and $B'$ are on the same chain if $B<B'$, $B=B'$ or $B> B'$; and $B \nsim B'$ or $B$ and $B'$ are not on the same chain if there is no such chain.

Blocks are ordered as a tree with the genesis block as root. So any two blocks have a common ancestor but two blocks not on the same chain do not have a common descendant.

A vote $v$ for a block $B$ by a validator $V$ is a message signed by $V$ containing the blockhash of $B$ and meta information like the round numbers and the type of vote. 

We call a set $S$ of votes tolerant if the number of voters with more than one vote in $S$ is at most $f$. We say that $S$ has supermajority for a block $B$ if the set of voters with votes for blocks $\geq B$ has size at least $(n+f+1)/2$.

The $2/3$-GHOST function $g(S)$ takes a set $S$ of votes and returns the block $B$ with highest block number such that $S$ has a supermajority for $B$. If there is no such block, then it returns `nil`. (if $f \neq \lfloor (n-1)/3 \rfloor$, then this is a misnomer and we may change the name accordingly.)

Note that, if $S$ is tolerant, then we can compute $g(S)$ by starting at the genesis block and iteratively looking for a child of our current block with a supermajority, which must be unique if it exists. Thus we have:
\begin{lemma} \label{lem:ghost-monotonicity}
Let $T$ be a tolerant set of votes. Then
\begin{enumerate}
\item The above definition uniquely defines $g(T)$
\item If $S \subseteq T$ has $g(S) \neq$ nil, then $g(S) \leq g(T)$.
\item If $S_i \subseteq T$ for $1 \leq i \leq n$ then all non-nil $g(S_i)$ are on a single chain with head $g(T)$.
\end{enumerate}

\end{lemma}

Note that we can easily update $g(S)$ to $g(S \cup \{v\})$, by checking if any child of $g(S)$ now has a supermajority.

3 tells us that even if validators see different subsets of the votes cast in a given voting round, this rule may give them different blocks but all such blocks are in the same chain under this assumption. 

We say that it is possible for a set $S$ to have a supermajority for $B$ if $2f+1$ validators either vote for a block $\not \geq B$ or equivocate in $S$. Note that if $S$ is tolerant, it is possible for $S$ to have a supermajority for $B$ if and only if there is a tolerant $T \supseteq S$ that has a supermajority for $B$.


We say that it is impossible for any child of $B$ to have a supermajority in $S$ if $S$ has votes from at least $2f+1$ validators and it is impossible for $S$ to have a supermajority for each child of $B$ appearing on the chain of any vote in $S$. Again, provided $S$ is tolerant, this holds if and only if for any possible child of $B$, there is no tolerant $T \subseteq S$ that has a supermajority for that child.

Note that it is possible for an intolerant $S$ to both have a supermajority for $S$ and for it to be impossible to have such a supermajority under these definitions, as we regard such sets as impossible anyway.

\begin{lemma} \label{lem:impossible}
\begin{itemize}
\item[(i)] If $B' \geq B$ and it is impossible for $S$ to have a supermajority for $B$, then it is impossible for $S$ to have a supermajority for $B'$.
\item[(ii)] If $S \subseteq T$ and it is impossible for $S$ to have a supermajority for $B$, then it is impossible for $T$ to have a supermajority for $B$.
\item[(iii)] If $g(S)$ exists and $B \nsim g(S)$ then it is impossible for $S$ to have a supermajority for $B$.
\end{itemize}
\end{lemma}

\section{The GRANDPA protocol}

We let $V_{r,v}$ and $C_{r,v}$ be the sets of prevotes and precommits respectively received by $v$ from round $r$ at the current time.

We define $E_{r,v}$, $v$'s estimate of what might have been finalised in round $r$, to be the last block in the chain with head $g(V_{r,v})$ that it is possible for $C_{r,r}$ to have a supermajority for. If either $E_{r,v} < g(V_{r,v})$ or it is impossible for $C_{r,v}$ to have a supermajority for any children of $g(V_{r,v})$,, then we say that ($v$ sees that) round $r$ is completable. $E_{0,v}$ is the genesis block (if we start at $r=1$).

We have a time bound $T$, that we hope is enough to send messages and gossip them to everyone. 

In round $r$ an honest validator $v$ does the following:

\noindent \fbox{\parbox{6.3in}{

\begin{enumerate}
\item $v$ can start round $r > 1$ when round $r-1$ is completable and $v$ has cast votes in all previous rounds where they are a voter. Let $t_{r,v}$ be the time we start round $r$.

\item At time $t_{r,v}$, if $v$ is the primary of this round and has not finalised $E_{r-1,v}$ then they broadcast $E_{r-1,v}$. If thy have finalised it, they can broadcast $E_{r-1,v}$ anyway (but don't need to).

\item If $v$ is a voter for the prevote of round $r$, $v$ waits until either it is at least time $t_{r,v}+2T$ or round $r$ is completable, then broadcasts a prevote. They prevote for the head of the best chain containing $E_{r-1,v}$ unless we received a block $B$ from the primary and $g(V_{r-1,v}) \geq B > E_{r-1,v}$, in which case they use the best chain containing $B$ instead.

\item If $v$ is a voter for the precommit step in round $r$, then they wait until $g(V_{r,v}) \geq E_{r-1,v}$ and one of the following  conditions holds
\begin{itemize}
\item[(i)] it is at least time $t_{r,v}+4T$, 
\item[(ii)] round $r$ is completable or
\item[(iii)] it is impossible for $V_{r,v}$ to have a supermajority for any child of $g(V_{r,v})$,
and then broadcasts a precommit for $g(V_{r,v})$ {\em( (iii) is optional, we can get away with just (i) and (ii))}.
\end{itemize}
\end{enumerate}

}}


\subsection{Finalisation}

If, for some round $r$, at any point after the precommit step of round $r$, we have that $B=g(C_{r,v})$ is later than our last finalised block and $V_{r,v}$ has a supermajority, then we finalise $B$. We may also send a commit message for $B$ that consists of $B$ and a set of precommits for blocks $\geq B$ (ideally for $B$ itself if possible see "Alternatives to the last blockhash" below). 

To avoid spam, we only send commit messages for $B$ if we have not receive any valid commit messages for $B$ and its descendants and we wait some time chosen uniformly at random from $[0,1]$ seconds or so before broadcasting.

If we receive a valid commit message for $B$ for round $r$, then it contains enough precommits to finalise $B$ itself if we haven't already done so, so we'll finalise $B$ as long as we are past the precommit step of round $r$.

\section{ Analysis }

\subsection{ Accountable Safety}

The first thing we want to show is asynchronous safety if we have at most $f$ Byzantine validators:

\begin{theorem} If the protocol finalises any two blocks $B,B'$ that have valid commit messages sent are on the same chain, then there are at least $f+1$ Byzantine voters who all voted in a particular vote. Furthermore, there is a synchronous procedure to find such a set.
\end{theorem}

The challenge procedure works as follows: If $B$ and $B'$ are committed in the same round, then the union of their precommits must contain at least $f$ equivocations, so we are done. Otherwise $B$ was committed in round $r$ and $B'$ in round $r' > r$ say. Then we ask the at least $n-f$ validators who precomitted $\geq B'$ in round $r$ in the commit message, why they precomitted.

We ask queries of the following form: 

- Why was $E_{r''-1} \not\geq B$  when you prevoted for or precomitted to $B'' \not\geq B$ in round $r'' > r$?

Which any honest validator should be able to respond to as is shown in Lemma \ref{lem:honest-answer} below. 

The response is of the following form:

- A either a set $S$ of prevotes for round $r''-1$ or a set $S$ of precommits for round $r''-1$  or  such that it is impossible for $S$ to have a supermajority for $B$.

If no validator responds, then we  have $n-f$ Byzantine validators. If any do, then if $r'' > r+1$,  we can ask the same query for $n-f$ validators in round $r''-1$.

If any responded and $r''=r+1$, then we have either a set $S$ of prevotes or precommits in round $r$ that it is impossible for $S$ to have a supermajority for $B$ in round $r$.

If $S$ is a set of precommits, then if we take the union of $S$ and the set of precommits in the commit message for $B$, then the resulting set of precommits for round $r$ has a supermajority for $B$ and it is impossible for it to have a supermajority for $B$. This is possible if the set is not tolerant and so there must be at least $f+1$ voters who equivocate an so are Byzantine.

If we get a set $S$ of prevotes for round $r$ that does not have a supermajority for $B$, then we need to ask a query of the form

- Which prevotes for round $r$ have you seen?

to all the voters of precommit in the commit message for $B$  who voted for blocks $B'' \geq B$. There must be $n-f$ such validators and a valid response to this query is a set $T$ of prevotes for round $r$ with a supermajority for $B''$ and so a supermajority for $B$.

If any give a valid response, by a similar argument to the above, $S \cup T$ will have $f+1$ equivocations.

So we either discover $f+1$ equivocations in a vote or else $n-f > f+1$ voters fail to validly respond like a honest voter could do to a query.


\begin{lemma} \label{lem:honest-answer}
An honest validator can answer the first type of query.
\end{lemma}
We first need to show that for any prevote or precommit in round $r$ cast by an honest validator $v$ for a block $B''$, at the time of the vote we had $B'' \geq E_{r-1,v}$. Prevotes should be for the head of a chain containing either $E_{r-1,v}$ or $g(V_{r,v})$ Since $g(V_{r,v}) \geq E_{r-1},v$, in either case we have $B'' \geq E_{r-1,v}$. Precommits should be for $g(V_{r,v})$ but $v$ waits until $g(V_{r,v}) \geq E_{r-1,v}$ before precommiting so again this holds.
Thus if $B'' \not\geq B$, then we had $E_{r-1,v} \not\geq B$.
Next we need to show that if we had $E_{r-1,v} \not\geq B$ at the time of the vote then we can respond to the query validly. 
If $B$ wasn't on the same chain with $g(V_{r-1,v})$, then by Lemma \ref{lem:impossible} (iii), it was impossible for $V_{r-1,v}$ to have a supermajority for $B$. If it was on the same chain as $g(V_{r,-1v})$, then it was on the same chain as $E_{r-1,v}$ as well.
Since $E_{r-1,v} \not\geq B$, in this case we must have $B > E_{r-1,v}$. However, possibly using that round $r-1$ is completable, it was impossible for $C_{r-1,v}$ to have a supermajority for any child of $E_{r-1,v}$ on the same chain with $g(V_{v,r})$ and in particular for the child of $E_{r-1,v}$ on $\textrm{chain}(B)$. By Lemma \ref{lem:impossible} (i), this means $C_{r-1,v}$ did not have a supermajority for $B$.

Thus we have that, at the time of th vote, for one of $V_{r-1,v}$, $C_{r-1,v}$, it was impossible to have a supermajority for $B$. The current sets $V_{r-1,v}$ and $C_{r-1,v}$ are supersets of those at the time of the vote, and so by Lemma \ref{lem:impossible} (ii), it is still impossible. Thus $v$ can respond validly.


This is enough to show Theorem 1. Not that if $v$ sees a commit message for a block $B$ in round $r$ and has that $E_{r',v} \not\geq B$, for some completable round $r' \geq r$, then they should also be able to start a challenge procedure that successfully identifies at least $f+1$ Byzantine voters in some round. Thus we have that:

\begin{lemma} \label{lem:overestimate-final}
If there at most $f$ Byzantine voters in any vote, $B$ was finalised in round $r$ and an honest participant $v$ sees that round $r' \geq r$ is completable, then $E_{r',v} \geq B$.
\end{lemma}

\subsection{Liveness }

We show the protocol is deadlock free and also that it finalises new blocks quickly in a weakly synchronous model.

Let's define $V_{r,v,t}$ be the set $V_{r,v}$ at time $t$ and similarly for $C_{r,v,t}$ and the block $E_{r,v,t}$ .


\begin{lemma} \label{lem:message-monotonicity-completed-estimate}
 Let $v,v'$ be (possibly identical) honest participants, $t,t'$ be times and $r$ be a round. Then if $V_{r,v,t} \subseteq V_{r,v',r'}$ and $C_{r,v,t} \subseteq C_{r,v',r'}$, all these sets are tolerant and $v$ sees that $r$ is completable at time $t$, then $E_{r,v,t} \leq E_{r,v',t'}$ and $v'$ sees that $r$ is completable at time $t'$.
\end{lemma}

\begin{proof} Since $v$ sees that $r$ is completable at time $t$, $V_{r,v,t}$, $C_{r,v,t}$ each contain votes from $n-f$ voters and so the same holds for $V_{r,v',t'}$ and $C_{r,v',t'}$. 
By Lemma \ref{lem:ghost-monotonicity}, $g(V_{r,v',t'}) \geq g(V_{r,v,t})$.
 Using Lemma \ref{lem:impossible}, since it is impossible for $C_{r,v,t}$ to have a supermajority for any children of $g(V_{r,v,t}$, it is impossible for $C_{r,v',t'})$ as well and so $E_{r,v',t'} \leq g(V_{r,v,t})$. But now $E_{r,v,t}$,$E_{r,v',t'}$ are the last blocks on $\textrm{chain}(g(V_{r,v,t}))$ that it is possible for $C_{r,v,t},C_{r,v',t'}$ respectively to have a supermajority for. Thus by Lemma \ref{lem:impossible} (ii), $E_{r,v',t'} \leq E_{r,v,t}$.
\end{proof}
 
\subsubsection{Deadlock Freeness}

Now we can show deadlock freeness for the asynchronous gossip network model, when a message that is sent or received by any honest participant is eventually received by all honest participants.

\begin{proposition} Suppose that we are in the asynchronous gossip network model and that at most $f$ voters for any vote are Byzantine. Then the protocol is deadlock free.\end{proposition}

\begin{proof} We need to show that if all honest participants reach some vote, then all of them eventually reach the next.

If all honest voters reach a vote, then they will vote and all honest participants see their votes. We need to deal with the two conditions that might block the algorithm even then. To reach the prevote of round $r$, a participant may be held up at the condition that round $r-1$ must be completable. To reach the precommit, a voter may be held up by the condition that $g(V_{r,v}) \geq E_{r-1,v}$.

For the first case, the prevote, let $S$ be the set of all prevotes from round $r-1$ that any honest voter saw before they precommitted in round $r-1$.
By Lemma \ref{lem:ghost-monotonicity}, when voter $v'$ precommitted, they do it for block $g(V_{r-1,v'}) \leq g(S)
$. Let $T$ be the set of precommits in round $r$ cast by honest voters. Then or any block $B \not\leq g(S)$, $T$ does not contain any votes that are $\geq B$ and so it is impossible for $T$ to have a supermajority for $B$. In particular, it is impossible for $T$ to have a supermajority for any child of $g(S)$. 

Now consider a voter $v$. By our network assumption, there is a time $t$ by which they have seen the votes in $S$ and $T$. Consider any $t' \geq t$. At this point we have $g(V_{r,v,t;}) \geq g(S)$. It is impossible for $C_{r,v,t'}$ to have a supermajority for any child of $g(S)$ and so $E_{r-1,v,t'} \leq g(S)$, whether or not this inequality is strict, we satisfy one of the two conditions for $v$ to see that round $r-1$ is completable at time $t'$. Thus if all honest voters reach the precommit vote of round $r-1$, all honest voters reach the prevote of round $r$.

Now we consider the second case, reaching the precommit. 
Note that any honest prevoter in round $r$ votes for a block $B_v \geq E_{r-1,v,t_v}$ where $t_v$ is the time they vote. Now consider any honest voter for the precommit $v'$. By some time $t'$, they have received all the messages received by each honest voter $v$ at time $t_v$ and $v'$'s prevote. 
Then by Lemma \ref{lem:overestimate-final}, $B_v \geq E_{r-1,v,t_v} \geq E_{r-1,v',t'}$. Since $V_{r,v',t'}$ contains these $B_v$, $g(V_{r,v',t'}) \geq  E_{r-1,v',t'}$. Thus if all honest voters prevote in round $r$, eventually all honest voters precommit in round $r$.

An easy induction completes the proof of the proposition.
\end{proof}

\subsubsection{Weakly synchronous liveness}

Now we consider the weakly synchronous gossip network model. The idea that there is some global stabilisation time($\textrm{GST}$) such that any message received or sent by an honest participant at time $t$ is received by all honest participants at time $\max\{t,\textrm{GST}\}+T$.

Let $t_r$ be the first time any honest participant enters round $r$ i.e. the minimum over honest participants $v$ of $t_{r,v}$.

\begin{lemma} \label{lem:timings}
Assume the weakly synchronous gossip network model and that each vote has at most $f$ Byzantine voters. Then if $t_r \geq \textrm{GST}$, we have that
\begin{itemize}
\item[(i)] $t_r \leq t_{r,v} \leq t_r+T$ for any honest participant $v$,
\item[(ii)] no honest voter prevotes before time $t_r+2T$,
\item[(iii)] any honest voter $v$ precommits at the latest at time $t_{r,v}+4T$,
\item[(iv)] for any honest $v$, $t_{r+1,v} \leq t_r + 6T$.
\end{itemize}
\end{lemma}


\begin{proof} Let $v'$ be one of the first validators to enter round $r$ i.e. with $t_{r,v'}=t_r$. 
By our network assumption, all messages received by $v'$ before they ended are received by all honest participants before time $t_r+T$.
 In particular at time $t_r$, $v'$ sees that all previous rounds are completable and so by Lemma \ref{lem:overestimate-final}, so does every other honest validator by time $t_r+T$. Also since for $r' < r$, at some time $s_{r'} \leq t_r$ $g(V_{r',v',s_r'}) \geq E_{r',v',s_r'}$, again by Lemma 4, for all honest $v$, $g(V_{r',v,t_r+T}) \geq E_{r',v,t_r+T}$. Looking at the conditions for voting, this means that any honest validator does not need to wait before voting in any round $r' \leq r$. 
 Thus they cast any remaining votes and enter round $r$ by time $t_r + T$. This shows (i).

For (ii), note that the only reason why an honest voter would not wait until time $t_{r,v}+2T \geq t_r+ 2T$ is when $n-f$ voters have already prevoted. But since some of those $n-f$ votes are honest, this is impossible before $t_r+2T$

Now an honest voter $v''$ prevotes at time $t_{r,v''}+2T \leq t_r +3T$ and by our network assumptions all honest validators receive this vote by time $t_r+4T$. An honest voter for the precommit $v$ has also received all messages that $v''$ received before they prevoted by then. Thus the block they prevoted has $B_{v''} \geq E_{r-1,v''} \geq E_{r-1,v,t_r+4T}$, since this holds for every honest voter $v''$, $g(V_{r,v,t_r+4T}) \geq E_{r-1,v,t_r+4T}$. Thus they will precommit by time $t_{r,v}+4T$ which shows (iii).

By the network assumption an honest voter $v'$'s precommit will be received by all honest validators $v$ by time $t_{r,v'}+ 5T \leq t_r+6T$. Since $v$ will also have received all prevotes $v$ say when they precommitted by this time, their vote $B_{v'}$ will have $B_{v'}=g(V_{r,v'}) \leq g(V_{r,v,t_r+6T})$. Thus $C_{r, v, t_r+6T}$ contains precommits from $n-f$ voters $v'$ with $B_{v'} \leq g(V_{r,v,t_r+6T})$ and thus it is impossible for $C_{r,v,t_r+6T}$ to have a supermajority for any children of $g(V_{r,v, t_r+6T})$. Thus $v$ sees that round $r$ is completable at time $t_r+6T$. Since they have already prevoted and precommitted if they were a voter, they will move to round $r+1$ by at latest $t_t+6T$. This is (iv).
\end{proof}

\begin{lemma} \label{lem:honest-prevote-timings}
Suppose $t_r \geq GST$ and very vote has at most $f$ Byzantine voters. Let $H_r$ be the set of prevotes ever cast by honest voters in round $r$. Then
\begin{itemize}
\item[(a)] any honest voter precommits to a block $\geq  g(H_r)$,

\item[(b)] every honest participant finalises $g(H_r)$ by time $t_r+6T$.
\end{itemize}
\end{lemma}

\begin{proof} For (a), we separate into cases based on which of the conditions (i)-(iii) that we wait for to precommit hold.

For (i), all honest voters prevote in round $r$ by time $t_r+3T$. So any honest voter $v$ who precommits at or after time $t_{r,v}+4T \geq t_r+4T$ has received all votes in $H_r$ and by Lemma \ref{lem:ghost-monotonicity}, precommits to a block $\geq g(H_r)$.

For (ii), we argue that no honest voter commits a block $\not\geq g(H_r)$ first. The result will then follow by an easy induction once the other cases are dealt with. Suppose that no honest voter has precommitted a block $\not \geq g(H_r)$ so far and that a voter $v$ votes early because of (ii).

Note that, since we assume that all precommits by honest voters so far were $\geq g(H_r)$, it is possible for $C_{r,v}$ to have a supermajority for $g(H_r)$. For (ii) to hold for a voter $v$ i.e for round $r$ to be completable, it must be the case that either it is impossible for $C_{r,v}$ to have a supermajority for $g(V_{r,v})$ or else be impossible for $C_{r,v}$ to have a supermajority for any children of $g(V_{r,v})$. By Lemma \ref{lem:impossible} cannot have $g(V_{r,v}) < g(H_r)$. But by Lemma \ref{lem:ghost-monotonicity}, these are on the same chain and so $g(V_{r,v}) \geq g(H_r)$. Since this is the block $v$ precommits to, we are done in case (ii)

For (iii), let $v$ be the voter in question. Note that since $n-f$ honest voters prevoted $\geq g(H_r)$, it is possible for $V_{r,v}$ to have a supermajority for $g(H_r)$. By Lemma \ref{lem:ghost-monotonicity}, $g(V_{r,v})$ is on the same chain as $g(H_r)$. For (iii), it is impossible for $V_{r,v}$ to have a supermajority for any children of $g(V_{r,v})$. If we had $g(V_{r,v}) < g(H_r)$, by Lemma \ref{lem:impossible}, this would mean that it would be impossible for $V_{r,v}$ to have a supermajority for $g(H_r)$ as well. So it must be that $g(V_{r,v} )\geq g(H_r)$ as required.

For (b), combining (a) and Lemma \ref{lem:timings} (iii), we have that any honest voter $v$ precommits $\geq g(H_r)$ by time $t_{r,v}+4T$. By our network assumption, all honest participants receive these precommits by time $t_r+6T$ and so finalise $g(H_r)$ if they have not done so already.
\end{proof}

\begin{lemma} \label{lem:primary-finalises}
 Suppose that $t_r \geq GST$, the primary $v$ of round $r$ is honest and no vote has more than $f$ Byzantine voters. Let $B=E_{r-1,v,t_{v,r}}$ be the block $v$ broadcasts if it is not final. Then every honest prevoter prevotes for the best chain including $B$ and all honest voter finalise $B$ by time $t_r+6T$.
 \end{lemma}

\begin{proof} By Lemma \ref{lem:timings} and our network assumptions, no honest voter  prevotes before time $t_r+2T \geq t_{r,v}+2T$ and so at this time, they will have seen all prevotes and precommits seen by $v$ at $t_{r,v}$ and the block $B$ if $v$ broadcast it then. By Lemma \ref{lem:message-monotonicity-completed-estimate}, any honest voter $v'$ has $E_{r-1,v'} \leq B \leq g(V_{r-1,v}$ then.

So if the primary broadcast $B$, then $v'$ prevotes for the best chain including $B$. If the primary did not broadcast $B$, then they finalise it. By Lemma \ref{lem:overestimate-final}, it must be that $E_{r-1,v'} \geq B$ and so $E_{r-1,v'}=B$ and so in this case $v'$ also prevotes for th best chain including $B$.

Since all honest voters prevote $\geq B$, $g(H_r) \geq B$ and so by Lemma \ref{lem:honest-prevote-timings}, all honest participants finalise $B$ by time $t_r+6T$
\end{proof}



\begin{lemma}
 Suppose that $t_r \geq GST+T$ and the primary of round $r$ is honest. 
Let $B$ be the latest block that is ever finalised in rounds  $<r$ (even if no honest participant finalises it until after $t_r$). If all honest voters for the prevote in round $r$ agree that the best chain containing $B$ include the same child $B'$ of $B$, then they all finalises some child of $B$ before $t_r+6T$.
\end{lemma}

\begin{proof} By Lemma \ref{lem:overestimate-final}, any honest participant sees that $E_{r-1} \geq B$ during round $r$. Let $v$ be the primary of round $r$ and $B''=E_{r-1,v,t_{r,v}}$. If $B'' > B$, then by Lemma \ref{lem:primary-finalises}, all honest validators finalise $B''$ by time $t_r+6T$ which means they finalised a child of $B$. If $B''=B$, then by Lemma \ref{lem:honest-prevote-timings}, all honest voters prevote for th best chain including $B$. By assumption these chains include $B'$ and so $g(H_r) \geq B$. By Lemma \ref{lem:honest-prevote-timings}, this means that $B'$ is finalised by time $t_r+6T$.
\end{proof}







\subsubsection{Recent Validity}

\begin{lemma} \label{lem:honest-recent-validity}
Suppose that $t_r \geq GST$, the primary of round $r$ is honest and all votes have at most $f$ Byzantine voters. Let $B$ be a block that no prevoter in round $r$ saw as being in the best chain of an ancestor of $B$ at the time they prevoted. Then either all honest validators finalise $B$ before time $t_r+6T$ or no honest validator ever has $g(V_{r,v}) \geq B$ or $E_{r,v} \geq B$.
\end{lemma}

\begin{proof} Let $v'$ be the primary of round $r$ and let $B'=E_{r-1,v',t_{r,v'}}$. If $B' \geq B$, then by Lemma \ref{lem:primary-finalises}, all honest validators finalise $B$ by time $t_r+6T$. If $B' \not\geq B$, then by Lemma \ref{lem:primary-finalises}, no honest validator prevotes $\geq B$ and so no honest validator ever has $g(V_{r,v}) \geq B$.
\end{proof}


\begin{corollary} For $t - 6T > t' \geq GST$, suppose that an honest validator finalises $B$ at time $t$ but that no honest voter has seen $B$ as in the best chain containing some ancestor of $B$ in between times $t'$ and $t$, then at least $(t-t')/6T - 1$ rounds in a row had Byzantine primaries. \end{corollary}




\section{Practicalities}

\subsection{Changing the voter set on-chain in an asynchronously safe way}

\subsubsection{Changing the voter set in an asynchronously safe way}

Suppose we have an on-chain protocol that decides we need a different voter set. Once everyone finalises the block, they know that we need to change the set. The protocol can cope with changing the voter set from some round $r$. The main difficulty is that the chain has no idea what the current round number is and even if we have a block that instructs us to change the voter set at round $r$, we might only finalise the block after round $r$. So instead we will not take advantage of the ability to change set from one round to the next. 

A block $B$ can contain an instruction that we should change to the voter set to some other set after some integer $m \geq 0$ blocks. If our best chain for a prevote contains such a block $B$, then we do not prevote for more than $m$ blocks after $B$, even if our best chain is longer. Thus if the current voter set has $n-f$ honest voters, they will only finalise $m$ blocks after such a $B$. We only accept votes and commit messages up top $m$ blocks after $B$ from the current set of validators.  

When some block $B'$ that is $m$ blocks after $B$ has been finalised, then the new validator set starts again at round $1$ with $E_{0}=B'$. Votes will need to contain additional metadata that indicates the validator set somehow.

\subsubsection{Unsafe fallback for changing the voter set after stalling}

In extreme circumstances, we may need to deal with $1/3$ of voters being offline. There is no asynchronously safe way of doing this. It also breaks the chain of signed statements by the existing set of voters saying who the future set of voters should be. And it means we may be vulnerable to being cut of by Byzantine participants. However if we are in a state when many voters go offline but the network is not partitioned, then we want a way to agree on a set of new voters to restart the finality gadget.

Every 100 blocks or so, we should put a valid commit message on chain. Honest block producers should put the most recent message on the chain, provided that there is one for a more recent block than 100 blocks ago. Then if a participant sees that their best chain has not had such a message for 1000 blocks and are not aware of any more recent blocks being finalised, then they set a new validator set to be one determined by the 900th block since the last commit message on chain. 

The protocol for selecting voters should require recent messages on chain signed by those voters so that this is likely to give a set of voters very few of whom are offline.

We should consider having to manually approve finality agreed upon by this new set to alleviate the security concerns above. But this still gives a way to canonically agree on a new set, in the event of WW3 or bad initialisation of a new chain.

\subsection{Alternatives to the last block hash}

The danger with voting for the last blockhash in the best chain is that maybe no one else will have seen and processed the next block. It would also be nice to make the most of BLS multisig/aggregation, which allows a single signature for many messages/signers than can be checked in time proportional to the number of different messages signed. 

To get round the first alone, it might be better to vote for a block 3/4 along (rounding further) the unfinalised chain , rather than for th head.

But the second suggests that maybe we should be including signatures for several of the latest blocks in a chain. We could include that last 2 or 3. We could also do e.g. the the blocks with block numbers with the last 2 multiples of each power of two since th last finalised block, which gives log unfinalised chain length messages but should have many blocks in common.

When presented with a vote that includes many blocks, we should interpret them as being for th last block we've seen if any. Then we need to be able to update that vote to a later block when that is seen. This retains monotonicity of a supermajority for/ it is impossible to have a supermajority for over time.

It does not matter if some of the votes are for a block that does not exist as everyone will ignore that part of the vote. But including votes for block that are seen but are not on a chain is an equivocation and is slashable. We need to count such votes as votes for the had of every chain in the vote (as someone might interpret them as for any one of them).

Then if we need to BLS aggregate votes that are $\geq B$ for a commit message or query response, it is OK to use any vote that is $\geq B$, not necessarily the vote for th head. This should reduce the number of blockhashs sign, in the optimistic case down to 1.

\subsection{  Block production rule}

If we adopt that rule that block producers should build on the best chain including the last finalised block, then if we don't finalise another block this will eventually include some prefix  beyond the last finalised block, and therefore the protocol is live by Lemma \ref{lem:honest-recent-validity}.

But the issue is that if agreement is much slower than block production, then we might have a prevote for a short chain on the last finalised block, then the best chain does not include that block and we build a long chain that is eventually never finalised. This could be fixed by building on $E_{r-1}$ or $E_r$. But if we do that, and these change very quickly, then we may never come to agreement on the best chain. 

So we have two possible chain selection rules for block producers:

\begin{enumerate}
\item Build on the best chain including the last finalise block B.
\item Build on best chain including whichever of $\{E_r,E_{r-1},B\}$ is latest and $\geq B$.
\end{enumerate}

1 is better if finalisation is happening quickly compared to block production and 2 is best if block production is much faster. We could also consider hybrid rules like adopt 1 unless we see that the protocol is stuck or slow, then we switch to 2.

\section{Why?}

\subsection{Why do we wait at the end of a round and sometimes before precommitting?}

If the network  is badly behaved, then these steps may involve waiting an arbitrarily long time. When the network is well behaved (after the GST in our model), we should not be waiting. Indeed there is little point not waiting to receive 2/3 of voters' votes as we cannot finalise anything without them. But if the gossip network is not perfect, an some messages never arrive, then we may need to implement voters asking other voters for votes from previous rounds in a similar way to the challenge procedure, to avoid deadlock.

In exchange for this, we get the property that we do not need to pay attention to votes from before the previous round in order to vote correctly in this one. Without waiting, we could be in a situation where we might have finalised a block in some round r, but the network becomes unreliable for many rounds and gets few votes on time, in which case we' need to remember the votes from round r to finalise the block later. 

\subsection{Why have a primary?}

We only need the primary for liveness. We need some form of coordination to defeat the repeated vote splitting attack. The idea behind that attack is that if we are in a situation where almost 2/3 of voters vote for something an the rest vote for another, then the Byzantine voters can control when we see a supermajority for something. If they can carefully time this, they may be able to split the next vote. Without the primary, they could do this for prevotes, getting a supermajority for a block $B$ late, then split precommits so we don't see that it is impossible for there to be a supermajority for $B$ until late. If $B$ is not the best block given the last finalised block but $B'$  with the same block number, they could stop either from being finalised like this even if the (unknown) fraction of Byzantine players is small.

When the network is well-behaved, an honest primary can defeat this attack by deciding how much we should agree on. We could also use a common coin for the same thing, where people would prevote for either the best chain containing $E_{r-1,v}$ or $g(V_{r-1,v})$ depending on the common coin. With on-chain voting, it is possible that we could use probabilistic finality of the block production mechanism - that if we don't finalise a block and always build on the best chain containing the last finalised block then not only will the best chain eventually converge, but if a block is behind the head of the best chain, then with positive probability, it will eventually be in the best chain everyone sees.

In our setup, having a primary is the simplest option for this.

\section{The asynchronous finality gadget problem}

Here we give an extension of the \cite{flp} result that shows the impossibility of having an asynchronous and deterministic finality gadget protocol and give an asynchronous protocol that uses a common coin primitive.

\subsection{Impossibility of a deterministic protocol} \label{ssec:impossibility}

The asynchronous binary fault tolerant agreement problem is as follows:

We have  number of voters which each have an initial $v_i$ in $\{0,1\}$

We may have one or more faulty nodes, which here means going offline at some point. Nodes have asynchronous communication - so any message arrives but we have no guarantee when it will.
The goal is to have all non-faulty nodes output the same $v$, which must be $0$ if all inputs $v_i$ are $0$ and $1$ if all are $1$.

Fischer, Lynch and Paterson\cite{flp} showed that this is impossible if there is one faulty node.

The binary fault-tolerant finality gadget problem is similar, except now there is an oracle $A$ that any node can call at any time with the following properties:

either $A$ always outputs $x$ in $\{0,1\}$ to all nodes at all times
or else there is an $x$ in $\{0,1\}$ and
for each node $i$, there is a $T_i$ such that when $i$ calls $A$ before $T_i$. it gives $x$ but if it calls $A$ after $T_i$, it returns not $x$ .

and we want that if A never switches, then all non-faulty nodes output x. If A does switch then all non-faulty nodes should output the same thing, but it can be 0 or 1. 

Then this is also impossible, even for one faulty node, which just goes offline. Note that this generalises Byzantine agreement, since if we could each node $i$ could call $A$ once at the start and use the output as $v_i$. (For the multi-valued case, we will define the problem so that this reduction does not hold.)


\begin{proof}[Proof sketch] We follow the notation of  \cite{flp} and assume for a contradiction that we use a correct protocol. 
Let $r$ be a run of the protocol where $A$ gives $0$ all the time. Then by correctness $r$ decides $0$. Now we consider what can happen when $A$ switches to $1$ after each configuration in $r$. If it switches to $1$ at the start, then the protocol decides $1$. If we switch to $1$ when all node have already decided $0$, then we decide $0$.

We claim that some configuration in the run $r$, where there are two runs from it where $A$ is always $1$ that decide $0$ and $1$. We call such states $1$-bivalent.
To see this, assume for a contradiction that $r$ contains no such configurations. Then there is are successive configurations $C$,$C'$ such that if $A$ return $1$ in the future from $C$ then we always decide $0$ but from $C'$, we always decide $1$.
Let events be $(p,m,x)$ where node (processor/validator) $p$ receives message $m$ (which my be null) and executes some code where any calls to A return $x$ in $\{0,1\}$, then sends some messages. 
Then there is some event $(p,m,0)$ that when applied to $C$ gives $C'$. Now suppose that $p$ goes offline at $C$, then if $A$ always returns $1$ afterwards, then we still decide $1$. Thus there is a run $r'$ that starts at $C$ where $p$ tales no steps, $A$ always returns $1$ and all other nodes still output $1$.
But since $p$ takes no steps in $r'$, we can apply $r'$ after $(p, m, 0)$ and so we have that $C'$ has a run where $A$ always returns $1$ but decides $1$, which is a contradiction.

Now let $C$ be a $1$-bivalent configuration. We can follow the FLP proof to show that there is a run from $C$ for which $A$ always returns $1$, all messages are delivered but all configurations are 1-bivalent and so the protocol never decides. This completes the proof by contradiction that there is no correct protocol.
\end{proof}

\subsection{1/5 BFT finality gadget using a common coin}




In this section, we will assume the asynchronous gossip network model. By the previous impossibility result, we will need to use randomness to get a finality gadget in this model. We assume that we have access to a common coin protocol.


For every vote, We have $n$ voters , at most $f$ of which are Byzantine and $n = 5f+1$.  For a voter $v$, Let $V_{r,v}$, $C_{r,v}$ be the set of prevotes and precommits from round $r$ that $v$ has seen.

\begin{enumerate}
\item Everyone prevotes for the best chain including the block they were locked to last round.
\item Wait until $V_{v,r}$ contains prevotes from $n-f$ voters.
\item Precommit $g_{3/5}(V_{r,v})$
\item Call a precommit for $B$ justified if $B \leq g_{3/5}(V_{r,v})$ and if $B < g_{3/5}(V_{r,v})$ then the child $B'$ of $B$ on the chain of $g_{3/5}(V_{r,v})$ has that there are votes from $f+1$ voters in $V_{r,v}$ that are not $\geq B'$.
Wait until $C_{r,v}$ has justified precommits from $n-f$ voters. 
\item Call the common coin, $s_r$
\item If $s_r=1$, finalise $g_{4/5}(C_r)$
\item lock to $g_{(4-3s_r)/5}(C_r)$ for next round.
\end{enumerate}

The common coin is a (secure cryptographic implementation of) the following protocol. It does not return a coin until more than $4f+1$ voters (for the prevote vote in the next round in case of ambiguity) call it. It returns at the latest shortly after all honest voters call it.
 When it does, it returns an $s_r$ sampled uniformly from $\{0,1\}$, identical for all who called it, and before $4f+1$ called it, no-one has any information about the result.

Here $g_{t}(S)$ is the $t$-GHOST function defined as follows. We construct a chain starting with the genesis block and adding the child of the current block such that most voters have votes $\geq$ it until there are $nt$ or less votes for any child of the current block, when we return the current block.

The idea behind the proof of asynchronous liveness is that for a particular block $B'$, some value of the common coin, either all the honest voters who received $4/5$ of precommits before the common coin was decided lock to $B'$ or none do. If we had a fixed threshold for locking, an adversarial choice of the number of precommits for $B'$ or its descendants could lead to some voters locking to it and some not (and indeed there would be runs that do this indefinitely as this is how the impossibility result works for this type of algorithm.)

Firstly we note that much of the machinery of ? and ? carries over to the $1/5$ byzantine case.

\begin{lemma} \label{lem:ghost-monotonicity-general}
Let $T$ be a set of votes such that at most $f$ voters have multiple votes in $T$. Let $t \geq (n+f)/2n$ Then
\begin{enumerate}
\item The above definition uniquely defines $g_t(T)$.
\item If $S \subseteq T$ has $g_t(S) \neq$ nil, then $g_t(S) \leq g_t(T)$ for $t \geq (n+f)/2n$.
\item If $S_i \subseteq T$ for $1 \leq i \leq n$ then all non-nil $g_t(S_i)$ are on a single chain with head $g(T)$.
\item If $r \leq s$, then $g_r(T) \geq g_s(T)$.
\end{enumerate}
\end{lemma}
So with $n=5f+1$, $g_{3/5}$ is sufficient for uniqueness.

First we need to show that the protocol is deadlock free. As long as all honest voters prevote and precommit, any participant eventually sees prevotes and precomits from $n-f$ voters. We just need to show that honest prevotes are eventually seen as justified.
\begin{lemma} Suppose that an honest voter $v$ precommits $B$ in round $r$. If $V'_{r,v}$ is the set of prevotes they saw at the time they precommited and another participant $v'$ sees all these prevotes i.e. $V_{r,v'} \supseteq V'_{r,v}$, then $v'$ sees $v$'s precommit for $B$ as justified. \end{lemma}
\begin{proof} $v$ precommits $B = g_{3/5}(V'_{r,v})$. Since $V_{r,v'} \supseteq V'_{r,v}$, $B \leq g_{3/5}(V_{r,v'})$ by Lemma \ref{lem:ghost-monotonicity-general} 2. So we just need to show that if  $B < g_{3/5}(V_{r,v'})$, $V_{r,v'}$ contains votes from $f+1$ voters that are not $\geq B'$ where $B'$ is the child of $B$ in the chain of $g_{3/5}(V_{r,v'})$. Since $B = g_{3/5}(V'_{r,v})$, from the definition of $g$, $B'$, like any child of $B$, does not have votes from $3f+1$ voters $\geq B'$ in $V'_{r,v}$. Since $V'_{r,v}$ contains votes from $4f+1$ voters, there are votes from at least $f+1$ voters that are $\not \geq B'$ in $V'_{r,v}$ and so also in $V_{r,v'}$.
\end{proof}

Our network assumption and a simple induction shows that we do not deadlock.
\begin{corollary} All honest voters eventually prevote and precommit in evrey round and all honest participants reach every round.\end{corollary}

\begin{lemma} \label{lem:possibly-final-implies-permanent-lock}
If there are enough precommits to finalise a block $B$ in round $r$, then all honest voters who prevote in future rounds will be locked to $B$ or its descendants when they do. At the end of the next round $r' > r$ with $s_{r'}=1$, all participants will have finalised $B$.
\end{lemma}
\begin{proof}
For $B$ to be finalised in round $r$, there need to be votes from more than $n-f$ voters that are $\geq B$ and $s_r=1$. Any honest participant $v$ also sees that $s_r=1$ and so they lock $g_{1/5}(C_{r,v})$. $C_{r,v}$ contains votes from at least $4f+1$ voters. At most $f$  voters can have votes  $\not\geq B$ in $C_{r,v}$ if they also voted $\geq B$ and at most $f$ voters do not have votes in $C_{r,v}$.
 Thus at least $2f+1$ voters have votes $\geq B$ in $C_{r,v}$. Because $g_{1/5}$ is not unique in general, to show that $g_{1/5}(C_{r,v}) \geq B$, we also need to show that no block $B' \nsim B$ has $f+1$ voters have votes $\geq B'$ in $C_{r,v}$. If this holds then the procedure to calculate $g_{1/5}$ will not follow chain that does not include $B$ and so it will return a block $\geq B$.
 Letting $V_r$ be the set of prevotes ever cast, note that any honest voter $v'$ prevotes for a block $g_{3/5}(V_{r,v'}) \leq g_{3/5}{V_r}$ and so as before honest voters precommit to blocks in one chain. Since many honest voters precommit $\geq B$, all precommit $\sim B$, and so if $f+1$ voters have votes $\geq B'$ in $B$ then since at least one of those are honest $B' \sim B$. Thus we have $g_{1/5}(C_{r,v}) \geq B$.

Since all honest voters prevote $\geq B$ in round $r+1$, any participant who waits for votes from $4f+1$ voters will see $g_{3/5}(V_{r+1}) \geq B$ and so all honest voters precommit $\geq B$ in round $r+1$. Since only at most $f$ validators vote $\not \geq B$, only precommits $\geq B$ are ever seen as justified by honest validators. Therefore all honest participants will see $g_{345}(C_{r+1}) \geq 
B$.
If $s_r=1$, this is enough to finalise $B$.Since $g_{1/5}(C_{r+1}) \geq g_{4/5}(C_{r+1}) \geq 
B$, whatever the common coin, all honest particupants lock $\geq B$. By induction, this holds for all future rounds.

\end{proof}

We want to show that this is asynchronously live:

\begin{proposition} Suppose that block $B$ is finalised before round $r$. With probability at least $1/2$ over the common coin in round $r$, if all validators agree that the best chain including the last finalised block $B$ includes a decedent $B''$, at the prevote step of rounds $r+1$ and $r+2$, then a descendant of $B$ is finalised the next time $s_r=1$ after round $r+2$ or earlier.
\end{proposition}

\begin{proof} By the Lemma \ref{lem:possibly-final-implies-permanent-lock}, all honest voters prevote in round $r$ for $B$ or its descendants and so all honest voters precommit to $B$ or its descendants.

Let $V_r$ be the set of prevotes of all voters.  Using Lemma \ref{lem:ghost-monotonicity-general}, all honest voters precommit $g_{3/5}(V_r)$ or its ancestors. Since some must precommit $\geq B$ for it to be finalised,  $g_{3/5}(V_r) \geq B$.

For the case $g_{3/5}(V_r)=B$, all honest voters precommit $B$ and so any honest participant sees that
$B = g_{1/5}(C_r) = g_{4/5}(C_r)$. Thus all honest participants
lock $B$ and so are free to prevote for $B''$ or its descendants in round $r+1$. Thus we finalise $B''$ in round $r+1$ or the next round when $s_r=1$ after that.

Otherwise, let $B'$ be the child of $B$ in the chain of $g_{3/5}(V_r)$. We seek to show that we finalise either $B'$ or $B''$.

Let $S$ be the set of honest voters who precommit in round $r$ before $4f+1$ voters call the common coin. Let $S'$ be the set of honest voters who call the common coin before it is decided. % Note that $S' \subset S$. 
Since $4f+1$  voters call the coin before it decided and honest voters who do so saw precommits from $4f+1$ voters, $S'$ and $S$ each contain at least $3f+1$ voters.

Let $h$ be the number of voters in $S$ that precommit $B'$ or its descendants. Note that the other $|S|-h$ voters just precommit $B$.

Now consider a particular voter $v$ and the set $C_{r,v}$ of precommits they received in step 4. the number of validators with precommits in  $C_{r,v}$ is at least $4f+1$. 
If $v \in S'$,
All the honest validators  
with precommits in $C_{r,v}$ are in $S$. In this case we have that the number of votes for $B'$ or its descendants in $C_{r,v}$, $m_v$ has $h-f \leq m_v < h+f$. For $v \notin S'$, since $f$ honest vali8dators can be outside $S$, we have $h-2f \leq m_v \leq h+2f$ 

Since any descendant of $B$ that is not $B'$ or its descendants receives less than $f$ precommits for it or its descendants, we have that either $g_{1/5}(C_{r,v})=B$ or $g_{1/5}(C_{r,v})\geq B'$ and similarly for $g_{4/5}(C_{r,v})$. Now note that if $h \geq 3f+1$, $m_v \geq f+1$ and so $g_{1/5}(C_{r,v}) \geq B'$. On the other hand if $h < 3f+1$, for $v \in S'$, $m_v < 4f+1$ and so $g_{4/5}(C_{r,v})=B$.

If $h \geq 3f+1$ and $s_r=1$, then every honest voter locks a block $\geq B'$. Thus is round $r+1$, they all prevote $\geq B'$.
 By similar reasoning to Lemma \ref{lem:possibly-final-implies-permanent-lock}, we finalise $B'$, the next round $r' > r$ that we have $s_{r'}=1$.
 
If $h < 3f+1$ and $s_r=0$, then every $v \in S'$ locks only $B$. But then all such $v$ will prevote their best chain containing $B$ and so a block $\geq B''$. There are only at most $2f$ voters who might not do this, the Byantine voters and the honest voters outside of $|S|$ who prevote $\geq B$. Thus any honest voter who has seen prevotes from $n-f$ voters either sees $g_{3/5}(V_{r+1,v})=B$ or $g_{3/5}(V_{r+1,v}) \geq B'$. Since all honest precommits are either $B$ or $\geq B''$, evry honest voter locks either $B$ or $\geq B''$. Since in round $r+2$, all honest voters see that the best chain including $B$ also includes $B''$, this time they all prevote $\geq B''$.  By similar reasoning to Lemma \ref{lem:possibly-final-implies-permanent-lock}, we finalise $B''$, by the next round $r' > r+1$ that we have $s_{r'}=1$.

Crucially note that $h$ depends only on $S$, which is determined when $4f+1$ voters call the common coin and before it is flipped. Thus $s_r$ is independent of $h$. If $h < 3f+1$ then $s_r=0$ with probability $1/2$ and if $h \geq 3f+1$ then $s_r=1$ with probability $1/2$. So with probability $1/2$, we have either both $h < 3f+1$ and $s_r=0$ or both $h \geq 3f+1$ and $s_r=1$. Thus with probability at least $1/2$, we finalise $B'$ or $B''$ before the next round after $r+1$ when $s_r=1$.
\end{proof}

\bibliography{grandpa}

\end{document}
